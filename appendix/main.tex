\chapter{Definitions}

\section{Contrapositive}\label{app:def:contrapositive}\date{26 March 2012}

It is often more useful to work with the contrapositive something than the original statement.

\begin{defn}
  The \textbf{contrapositive} of the statement \(P \to Q \) is the statement \(\neg Q \to \neg P\).
\end{defn}


\begin{ex}
  Find the contrapositive of the following:
  \begin{quote}
    ``If it is snowing then it is cold.''
  \end{quote}
  \begin{sol}
    \begin{quote}
      ``If it is snowing then it is cold.''\(\to\)``If NOT (it is cold) then NOT (it is snowing.)''
    \end{quote}
  \end{sol}
\end{ex}
\begin{ex}
  Find the contrapositive of the following:
  \begin{quote}
    ``If a number squared is odd, then the number itself is odd.''
  \end{quote}
  \[ n^2=2k+1\to n=2j+1 \]
  \begin{sol}
    ``If NOT (number itself is odd) then NOT (number squared is odd).''
    \[n=2k \to n^2 = 2j \]
    \begin{proof}
      \begin{align*}
        n &=2k+1 \\
        n^2 &= (2k+1)(2k+1) \\
        &= 4k^2+4k+1 \\
        &= 2(\text{integer})+1
      \end{align*}
    \end{proof}
  \end{sol}
\end{ex}

\chapter{Important Concepts}

\section{Quadratic Formula}
Quadratic formula\index{quadratic formula}
\begin{equation}
  x=\frac{-b\pm\sqrt{b^2-4ac}}{2a}
  \label{app:eq:quadratic}
\end{equation}

\section{Conjugate}\label{app:def:conjugate}

In algebra, the \textbf{conjugate}\index{conjugate} of a \emph{binomial} is another binomial formed by taking the opposite of the second term of the first binomial. For the initial binomial
\[ a + b\]
its conjugate would be
\[a - b.\]

Meanwhile, for the expression \[a^2+b^2\] we can factor this to produce \[(a-b)(a+b)\] where one expression is the conjugate of the other.

\chapter{Logic Circuits} \index{logic circuits}
\epigraph{
Look, everything we're putting into that box becomes ungrounded, and I don't
mean grounded like to the earth, I mean, not tethered. I mean, we're blocking
whatever keeps it moving forward and so they flip-flop. Inside the box it's like
a street, both ends are cul-de-sacs. I mean, this isn't frame dragging or
wormhole magic, this is basic mechanics and heat 101.}
{\emph{Primer}, 2004}

A logic circuit receives input signals \(p_1, p_2, \ldots, p_n\), each a bit, and produces output signals
\(s_1, s_2, \ldots, s_n\), each a bit.

\begin{figure}[h]
  \center{
      \subfigure[and]{
        \begin{circuitikz} \draw
          (0,2) node[and port] (myand1) {}
          (myand1.in 1) node[anchor=east] {}
          (myand1.in 2) node[anchor=east] {}
          (myand1.out) node[anchor=west] {};
        \end{circuitikz}
      }
      \subfigure[or]{
        \begin{circuitikz} \draw
          (0,2) node[or port] (myor1) {}
          (myor1.in 1) node[anchor=east] {}
          (myor1.in 2) node[anchor=east] {}
          (myor1.out) node[anchor=west] {};
        \end{circuitikz}
      }
      \subfigure[nor]{
        \begin{circuitikz} \draw
          (0,2) node[nor port] (mynor1) {}
          (mynor1.in 1) node[anchor=east] {}
          (mynor1.in 2) node[anchor=east] {}
          (mynor1.out) node[anchor=west] {};
        \end{circuitikz}
      }
      \subfigure[xor]{
        \begin{circuitikz} \draw
          (0,2) node[xor port] (myxor1) {}
          (myxor1.in 1) node[anchor=east] {}
          (myxor1.in 2) node[anchor=east] {}
          (myxor1.out) node[anchor=west] {};
        \end{circuitikz}
      }
      \subfigure[not]{
        \begin{circuitikz} \draw
          (0,2) node[not port] (mynot1) {}
          (mynot1.in) node[anchor=east] {}
          (mynot1.out) node[anchor=west] {};
        \end{circuitikz}
      }
  }
  \caption{Basic logic gates.}
\end{figure}

\begin{comment}
\begin{figure}[h]
  \begin{center}
    \begin{circuitikz}
      \draw
      (8, 2) node[and port] (and0) {}

      (3, 4) node[or port] (or0) {}
      (3, 0) node[or port] (or1) {}
      (5, 2) node[or port] (or2) {}

      (1, 4) node[not port] (not0) {}
      (2, 2) node[not port] (not1) {}
      (1, 0) node[not port] (not2) {}

      (not0.out) -- (or0.in 2)
      (not0.in) -- (0, 4) node[anchor=east] {\(p_2\)}
      (or0.in 1) -- (0, 5) node[anchor=east] {\(p_1\)}
      (not1.in) -- (0, 3) node[anchor=east] {\(p_3\)}
      (not2.in) -- (0, 0) node[anchor=east] {\(p_5\)}
      (or1.in) -- (0, 2) node[anchor=east] {\(p_4\)}
      (and0.out) -- (9,2) node[anchor=west] {\(s_1\)}
      (or1.out) -- (or2.in 2)
      (or0.out) -- (and0.in 1)
      (not1.out) -- (or2.in 1)
      (not2.out) -- (or1.in 2)
      (or2.out) -- (and0.in 2);
    \end{circuitikz}
  \end{center}
  \caption{A simple logic circuit}
\end{figure}
\end{comment}



\chapter{The Fibonacci Sequence and the Golden Ratio}
\setstretch{1.618}


\section{The Fibonacci Sequence}

\begin{figure}[h]
  \begin{center}
    \input{fibonacci/fibgraph1}
  \end{center}
  \caption{A plot of equation \ref{eq:nint}.}
\end{figure}
The \emph{Fibonacci Sequence} is the first recursive number sequence known in Europe. Its first 10 numbers are

\[ 1, 1, 2, 3, 5, 8, 13, 21, 34, 55\dots \quad \text{.} \]

A Fibonacci sequence, in general, is any sequence of numbers in which each number is the sum of the two preceeding numbers.\cite{britannica12}


\subsection{History}

French mathematician Edouard Lucas coined the term ``Fibonacci sequence'' in the 19th century.
The sequence is found throughout nature, as in the spirals of sunflower heads, pine cones, snail shells, and animal horns.\cite{britannica12}
Because of this natural prevalence, patterns based on the Fibonacci sequence are considered aesthetically pleasing.
The sequence can be found in Mozart and Beethoven's works as well as in classical art and architecture. \cite[p.~94]{design10}


\subsection{Mathematics}

\begin{defn}The \emph{Fibonacci numbers} are the sequence of numbers \(\{F_n\}^\infty_{n=1}\) defined by the linear recurrence equation
\begin{equation}
  F_n=F_{n-1}+F_{n-2} \text{.}
\end{equation}
Often, we will see them defined with \(F_0=0\).
\end{defn}

This can be represented in the \emph{closed form}
\begin{equation}
  F_n=\left[ \frac{\Phi^n}{\sqrt{5}}\right]
  \label{eq:nint}
\end{equation}
where \([x]\) is the \emph{nearest integer function}. \cite{mwfib}


\section{The Golden Ratio}

\begin{figure}[ht]
  \begin{center}
  \includegraphics{fibonacci/vitruvian.jpg}
  %\includegraphics[width=0.225\textwidth]{vitruvian.jpg}
  \end{center}
  \caption{The Vitruvian Man, said to depict ideal human proportions, bases its proportions on the golden ratio.\cite[p.~115]{design10}}
  %\footnote{\url{http://en.wikipedia.org/wiki/File:Da_Vinci_Vitruve_Luc_Viatour.jpg}
\end{figure}


The Fibonacci sequence and the golden ratio are closely related.

\begin{defn}
  The \emph{golden ratio}, denoted \( \Phi \), is given by the positive solution to the equation
\begin{equation}
  \Phi^2 - \Phi - 1 = 0
\end{equation}
\end{defn}

Using the quadratic equation
(\ref{app:eq:quadratic})
we can find that
\begin{align*}
  \Phi =& \frac{1 \pm \sqrt{1^2-4(1)(-1)}}{2} \\
  =& \frac{1 \pm \sqrt{1+4}}{2} \\
  =& \frac{1 \pm \sqrt{5}}{2} \\
  \intertext{and taking the positive root}
  \Phi =& \frac{1 + \sqrt{5}}{2} \\
  =& 1.6180339887498948\dots
\end{align*}
\cite{mwgolden}

We will notice that many closed-form representations of the Fibonacci sequence use the golden ratio.
For example, \emph{Binet's Formula}
\begin{equation}
  F_n=\frac{\Phi^n-(-\Phi)^{-n}}{\sqrt{5}}
  \label{eq:binet}
\end{equation}
derived\footnote{Though not for the first time.} by Binet in 1843 and equation \ref{eq:nint} both write \(F_n\) in terms of \( \Phi \).\cite{mwbinet}

The ratio of consecutive terms in the Fibonacci sequence approximate the golden ratio:
\begin{align*}
  \frac{1}{1} &= 1 \\
  \frac{2}{1} &= 2 \\
  \frac{3}{2} &= 1.5 \\
  \frac{5}{3} &= 1.\overline{6}\dots \\
  \frac{8}{5} &= 1.6 \\
  \frac{13}{8} &= 1.625 \\
  \frac{21}{13} &\approx 1.6153846
\end{align*}
Through this, we can conclude that
\begin{equation}
  \lim_{n\to \infty} \frac{F_n}{F_{n-1}}=\Phi
  \label{eq:limphi}
\end{equation}
\cite{mwfib}

\begin{figure}[h]
  \begin{center}
    % GNUPLOT: LaTeX picture
\setlength{\unitlength}{0.240900pt}
\ifx\plotpoint\undefined\newsavebox{\plotpoint}\fi
\sbox{\plotpoint}{\rule[-0.200pt]{0.400pt}{0.400pt}}%
\begin{picture}(1500,900)(0,0)
\sbox{\plotpoint}{\rule[-0.200pt]{0.400pt}{0.400pt}}%
\put(130.0,82.0){\rule[-0.200pt]{315.338pt}{0.400pt}}
\put(110,82){\makebox(0,0)[r]{ 0}}
\put(130.0,82.0){\rule[-0.200pt]{4.818pt}{0.400pt}}
\put(130.0,179.0){\rule[-0.200pt]{315.338pt}{0.400pt}}
\put(110,179){\makebox(0,0)[r]{ 0.5}}
\put(130.0,179.0){\rule[-0.200pt]{4.818pt}{0.400pt}}
\put(130.0,276.0){\rule[-0.200pt]{315.338pt}{0.400pt}}
\put(110,276){\makebox(0,0)[r]{ 1}}
\put(130.0,276.0){\rule[-0.200pt]{4.818pt}{0.400pt}}
\put(130.0,373.0){\rule[-0.200pt]{315.338pt}{0.400pt}}
\put(110,373){\makebox(0,0)[r]{ 1.5}}
\put(130.0,373.0){\rule[-0.200pt]{4.818pt}{0.400pt}}
\put(130.0,471.0){\rule[-0.200pt]{315.338pt}{0.400pt}}
\put(110,471){\makebox(0,0)[r]{ 2}}
\put(130.0,471.0){\rule[-0.200pt]{4.818pt}{0.400pt}}
\put(130.0,568.0){\rule[-0.200pt]{315.338pt}{0.400pt}}
\put(110,568){\makebox(0,0)[r]{ 2.5}}
\put(130.0,568.0){\rule[-0.200pt]{4.818pt}{0.400pt}}
\put(130.0,665.0){\rule[-0.200pt]{315.338pt}{0.400pt}}
\put(110,665){\makebox(0,0)[r]{ 3}}
\put(130.0,665.0){\rule[-0.200pt]{4.818pt}{0.400pt}}
\put(130.0,762.0){\rule[-0.200pt]{214.160pt}{0.400pt}}
\put(1419.0,762.0){\rule[-0.200pt]{4.818pt}{0.400pt}}
\put(110,762){\makebox(0,0)[r]{ 3.5}}
\put(130.0,762.0){\rule[-0.200pt]{4.818pt}{0.400pt}}
\put(130.0,859.0){\rule[-0.200pt]{315.338pt}{0.400pt}}
\put(110,859){\makebox(0,0)[r]{ 4}}
\put(130.0,859.0){\rule[-0.200pt]{4.818pt}{0.400pt}}
\put(130.0,82.0){\rule[-0.200pt]{0.400pt}{187.179pt}}
\put(130,41){\makebox(0,0){ 0}}
\put(130.0,82.0){\rule[-0.200pt]{0.400pt}{4.818pt}}
\put(294.0,82.0){\rule[-0.200pt]{0.400pt}{187.179pt}}
\put(294,41){\makebox(0,0){ 1}}
\put(294.0,82.0){\rule[-0.200pt]{0.400pt}{4.818pt}}
\put(457.0,82.0){\rule[-0.200pt]{0.400pt}{187.179pt}}
\put(457,41){\makebox(0,0){ 2}}
\put(457.0,82.0){\rule[-0.200pt]{0.400pt}{4.818pt}}
\put(621.0,82.0){\rule[-0.200pt]{0.400pt}{187.179pt}}
\put(621,41){\makebox(0,0){ 3}}
\put(621.0,82.0){\rule[-0.200pt]{0.400pt}{4.818pt}}
\put(785.0,82.0){\rule[-0.200pt]{0.400pt}{187.179pt}}
\put(785,41){\makebox(0,0){ 4}}
\put(785.0,82.0){\rule[-0.200pt]{0.400pt}{4.818pt}}
\put(948.0,82.0){\rule[-0.200pt]{0.400pt}{187.179pt}}
\put(948,41){\makebox(0,0){ 5}}
\put(948.0,82.0){\rule[-0.200pt]{0.400pt}{4.818pt}}
\put(1112.0,82.0){\rule[-0.200pt]{0.400pt}{162.607pt}}
\put(1112.0,839.0){\rule[-0.200pt]{0.400pt}{4.818pt}}
\put(1112,41){\makebox(0,0){ 6}}
\put(1112.0,82.0){\rule[-0.200pt]{0.400pt}{4.818pt}}
\put(1275.0,82.0){\rule[-0.200pt]{0.400pt}{162.607pt}}
\put(1275.0,839.0){\rule[-0.200pt]{0.400pt}{4.818pt}}
\put(1275,41){\makebox(0,0){ 7}}
\put(1275.0,82.0){\rule[-0.200pt]{0.400pt}{4.818pt}}
\put(1439.0,82.0){\rule[-0.200pt]{0.400pt}{187.179pt}}
\put(1439,41){\makebox(0,0){ 8}}
\put(1439.0,82.0){\rule[-0.200pt]{0.400pt}{4.818pt}}
\put(130.0,82.0){\rule[-0.200pt]{0.400pt}{187.179pt}}
\put(130.0,82.0){\rule[-0.200pt]{315.338pt}{0.400pt}}
\put(1279,819){\makebox(0,0)[r]{\(F_n / F_{n-1}\)}}
\put(1299.0,819.0){\rule[-0.200pt]{24.090pt}{0.400pt}}
\put(130,82){\usebox{\plotpoint}}
\multiput(130.58,82.00)(0.500,2.147){325}{\rule{0.120pt}{1.815pt}}
\multiput(129.17,82.00)(164.000,699.234){2}{\rule{0.400pt}{0.907pt}}
\multiput(294.58,779.81)(0.500,-1.440){323}{\rule{0.120pt}{1.251pt}}
\multiput(293.17,782.40)(163.000,-466.404){2}{\rule{0.400pt}{0.625pt}}
\multiput(457.00,316.58)(0.701,0.499){231}{\rule{0.661pt}{0.120pt}}
\multiput(457.00,315.17)(162.629,117.000){2}{\rule{0.330pt}{0.400pt}}
\multiput(621.00,431.92)(1.647,-0.498){97}{\rule{1.412pt}{0.120pt}}
\multiput(621.00,432.17)(161.069,-50.000){2}{\rule{0.706pt}{0.400pt}}
\multiput(785.00,383.58)(4.611,0.495){33}{\rule{3.722pt}{0.119pt}}
\multiput(785.00,382.17)(155.274,18.000){2}{\rule{1.861pt}{0.400pt}}
\multiput(948.00,399.93)(12.468,-0.485){11}{\rule{9.471pt}{0.117pt}}
\multiput(948.00,400.17)(144.342,-7.000){2}{\rule{4.736pt}{0.400pt}}
\multiput(1112.00,394.61)(36.184,0.447){3}{\rule{21.833pt}{0.108pt}}
\multiput(1112.00,393.17)(117.684,3.000){2}{\rule{10.917pt}{0.400pt}}
\put(1275,395.67){\rule{39.508pt}{0.400pt}}
\multiput(1275.00,396.17)(82.000,-1.000){2}{\rule{19.754pt}{0.400pt}}
\put(1279,778){\makebox(0,0)[r]{\( \Phi \)}}
\multiput(1299,778)(20.756,0.000){5}{\usebox{\plotpoint}}
\put(1399,778){\usebox{\plotpoint}}
\put(130,396){\usebox{\plotpoint}}
\multiput(130,396)(20.756,0.000){8}{\usebox{\plotpoint}}
\multiput(294,396)(20.756,0.000){8}{\usebox{\plotpoint}}
\multiput(457,396)(20.756,0.000){8}{\usebox{\plotpoint}}
\multiput(621,396)(20.756,0.000){8}{\usebox{\plotpoint}}
\multiput(785,396)(20.756,0.000){8}{\usebox{\plotpoint}}
\multiput(948,396)(20.756,0.000){8}{\usebox{\plotpoint}}
\multiput(1112,396)(20.756,0.000){8}{\usebox{\plotpoint}}
\multiput(1275,396)(20.756,0.000){8}{\usebox{\plotpoint}}
\put(1439,396){\usebox{\plotpoint}}
\put(130.0,82.0){\rule[-0.200pt]{0.400pt}{187.179pt}}
\put(130.0,82.0){\rule[-0.200pt]{315.338pt}{0.400pt}}
\end{picture}

  \end{center}
  \caption{Equation \ref{eq:limphi} converges to \( \Phi \).}
\end{figure}


\section{Culture}

\subsection{Lateralus}

The song ``\emph{Lateralus}'' by the American rock band Tool counts out the Fibonacci sequence in its syllables:\footnote{\url{http://en.wikipedia.org/w/index.php?title=Lateralus\%20(song)&oldid=479876017}}
\begin{figure}[H]
\begin{tabular}{r|l}
1 & Black,                 \\
1 & then,                  \\
2 & white are,             \\
3 & all I see,             \\
5 & in my infancy,       \\
8 & red and yellow then came to be,             \\
5 & reaching out to me,   \\
3 & lets me see.           \\
2 & There is,              \\
1 & so,                    \\
1 & much,                  \\
2 & more and               \\
3 & beckons me,           \\
5 & to look through to these,                    \\
8 & infinite possibilities.                \\
13 & As below so above and beyond I imagine,\\
8 & drawn beyond the lines of reason.\\
5 & Push the envelope. \\
3 & Watch it bend. \\
\end{tabular}
\caption{ Maynard James Keenan's vocals.}
\end{figure}

