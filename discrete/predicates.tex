\chapter{Predicates and Quantifiers}

Propositional logic is too simple for us to make many types of conclusions.
Instead, we use \textbf{predicate logic}, which allows us to make general
statements about objects and their properties.

A \textbf{propositional function}\index{propositonal function}, $P(x)$, is a
type of \textbf{predicate}\index{predicate} in
predicate logic. The important thing about propositional functions is that their
truth value depends on the value of a variable, $x$. A propositional function
becomes a proposition when a value is assigned to $x$.

\section{Universal Quantification}

If we wish to state that a given propositional function is true for all possible
values in a domain, we use the \emph{universal quantification} of that function.
Just as values for a variable must be stated in order for a propositional
function to have a truth value, a \textbf{domain of discourse}\index{domain of
discourse} must be specified
in addition to the universal quantification. This is often referred to as just
the \emph{domain}.

\begin{defn}
  The \textbf{universal quantification}\index{universal quantification} of $P(x)$ is the statement
  \begin{quote}
    ``$P(x)$ for all values of $x$ in the domain.''
  \end{quote}
  The notation $\forall x P(x)$ denotes the universal quantification of
  $P(x)$.%\cite[p.~40]{rosen}
  % This is really stupid to cite. It's public domain/original enough to be mine.
\end{defn}

To show that the universal quantification of $P(x)$ is false for a domain, find
an $x$ for which $P(x)$ is false.

\section{Existential Quantification}\index{existential quantification}

If we wish to state that an element exists in a domain, we use the
\emph{existential quantification} of a propositional function.

\begin{defn}
  The \textbf{existential quantificaiton}\index{existential quantification} of $P(x)$ is the proposition
  \begin{quote}
    ``There exists an element $x$ in the domain such that $P(x)$''
  \end{quote}
  We use the notation $\exists P(x)$ for the existential quantification of
  $P(x)$.%\cite[p.~42]{rosen}
  % see earlier comment for universal quantification.
\end{defn}

In order to show that the existential quantification of $P(x)$ is false, we must
show that $P(x)$ is false for every possible value of $x$ in the domain.

A specific case of existential quantification is defined by the
\textbf{uniqueness quantifier}\index{uniqueness quantifier}, $\exists!$ or $\exists_1$. The notation
\[ \exists! x P(x) \]
is the statement ``There exists a unique $x$ such that $P(x)$ is true.'' The
downside to the uniqueness quantifier is that the rules of inference for
existential quantification cannot be used on it. Since propositional logic can
be used to express uniqueness already, we should try to avoid use of uniqueness
quantification.

\section{Logical Equivalence of Quantified Propositions}\index{logical
equivalence}

The following is a quote from Rosen's \emph{Discrete Mathematics and its Applications} on the issue:

\begin{quote}
  Statements involving predicates and quantifiers are \textbf{logically
  equivalent} if and only if they have the same truth value no matter what
  predicates are substituted into the statements and which the domain of
  discourse is used for the variables in these propositional functions. We use the
  notation $S \equiv T$ to indicate that two statements $S$ and $T$ involving
  predicates and quantifiers are logically equivalent.

  \hfill\cite[p.~45]{rosen}
\end{quote}

\section{DeMorgan's Laws for Quantifiers}\index{DeMorgan's laws for quantifiers}

\begin{equation}
  \neg \exists x P(x) \equiv \forall x \neg P(x)
\end{equation}
\begin{equation}
  \neg \forall x P(x) \equiv \exists x \neg P(x)
\end{equation}

\section{Order of Quantifiers}\index{quantifiers, order of}

The quantificaiton
\begin{equation}
  \exists y \forall x Q(x, y)
\end{equation}
denotes the propositon
\begin{quote}
  ``There is a real number $y$ such that for every real number $x$, $Q(x, y)$.''
\end{quote}

The quantificaiton
\begin{equation}
  \forall x \exists y Q(x, y)
\end{equation}
denotes the propositon
\begin{quote}
  ``For every real number $x$ there is a real number $y$ such that $Q(x, y)$.''
\end{quote}

