\chapter{Complex Numbers}\label{ch:complex}
Earl A. Coddington, professor of mathematics at UCLA, offers an extremely helpful crash-course in complex numbers in his book \emph{An Introduction To Ordinary Differential Equations}, Chapter 0 \cite{coddington}.
Most of the initial knowledge in this chapter comes from my notes on that chapter, but I will attempt to provide pictures and examples where I found the source text lacking.
\begin{defn}
  A \keyword{complex number}{complex number} is an ordered pair of real numbers $(x, y)$.
  If $z$ is a complex number, we write
  \begin{equation}
    z = (x,y).
  \end{equation}
\end{defn}
\begin{defn}
  The \keyword{sum}{complex sum} $z_1+z_2$ is the complex number given by
  \begin{equation}
    z_1 + z_2 = (x_1 + x_2, y_1 + y_2).
    \label{eq:complexsum}
  \end{equation}
\end{defn}
\begin{defn}
  If $z=(x,y)$, the \keyword{negative}{negative} of $z$, denoted $-z$, is defined to be the number
  \begin{equation}
    -z = (-x, -y).
  \end{equation}
\end{defn}
\begin{defn}
  The \keyword{zero}{zero} complex number, written simply 0, is defined as
  \begin{equation}
    0 = (0, 0).
  \end{equation}
\end{defn}
Since \eref{eq:complexsum} defines complex sums in terms of just real number addition operations, and we know that these real number operations are commutative, it follows that
\begin{equation}
  z_1+z_2 = z_2 + z_1.
\end{equation}
Likewise does the associative property of addition for real numbers hold for complex numbers:
\begin{equation}
  (z_1 + z_2) + z_2 = z_1 + (z_2 + z_3).
\end{equation}
And the number $0$ provides our additive identity:
\begin{equation}
  z + 0 = z.
\end{equation}
Finally, we have an additive inverse for complex numbers
\begin{equation}
  z+(-z)=0.
\end{equation}
For additional information on these properties as they apply to the set of real numbers, I will direct the reader to Michael Spivak's \emph{Calculus, Third Edition}, perhaps the single greatest introduction to ``real mathematics'' ever written.
These properties, and their importance with regard to real numbers, is detailed extensively in the first chapter.
\begin{defn}
  The \keyword{difference}{difference}, $z_1-z_2$, is defined by
  \begin{equation}
    z_1-z_2 = z_1 + (-z_2).
  \end{equation}
\end{defn}
\begin{defn}
  The \keyword{product}{product} $z_1z_2$ is defined by
  \begin{equation}
    z_1z_2 = (x_1x_2 - y_1y_2, x_1 y_2 + x_2 y_1).
    \label{eq:complex_product}
  \end{equation}
\end{defn}
\begin{remark}
  \eref{eq:complex_product} can be found by performing basic multiplication on the following form of the numbers:
  \begin{align*}
    z_1 &= x_1 + \iu y_1 \\
    z_2 &= x_2 + \iu y_2 \\
    z_1z_2 &= (x_1+\iu y_1)(x_2 + \iu y_2)
  \end{align*}
  In order to use this, however, we must define the following units:
\end{remark}
\begin{defn}
  The \keyword{unit}{unit} complex number is the number $(1,0)$.
  This may be multiplied by any complex number $z=(x,y)$ and the product will always be $z$.
\end{defn}
\begin{defn}
  The \keyword{imaginary unit}{imaginary unit} is defined to be the number \[\iu = (0,1).\]
\end{defn}
From those definitions, we see that if $z=(x,y)$ we can write it in terms of its real and imaginary parts as follows:
\begin{align}
  z&=x(1,0)+y(0,1),
  \intertext{which is equivalent to stating}
  z&=x+\iu y.
\end{align}
