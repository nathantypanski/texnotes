
\chapter*{Preface}\epigraph{
  Hold to the now, the here, through which all future plunges to the
  past.
}
{Ulysses, James Joyce, Episode 9}
\addcontentsline{toc}{chapter}{Preface}

I started this notebook my Math 240 (Calculus II) class at Christopher Newport
University on \date{February 13, 2012}. I had picked up the basics of \LaTeX\ in my free hour
before class because I wanted to learn how to type mathematical documents. Why?
Because \LaTeX\ is cool.

I started to care about math because it serves as the logical foundation for
physics. I realized quickly in my studies that I did not know enough math, and I
did not know it rigorously enough to truly understand physics. I cared about
physics because it is a prerequisite for understanding the basics of computer
engineering, my major.
% thanks Kyle Martin for correcting a typo here.

% EDIT: REMOVED 11/6/2012
%%%%%%%%%%
% Discrete mathematics, a subject which has only recently grown to popularity in
% concurrence with computer science, is included first because it describes much
% of the logical foundation for mathematics in ways I had never encountered
% before. In many ways, it involves thinking about the basic thought processes
% that we take for granted in continuous mathematics. When we make claims in
% mathematics such as
% \begin{quote}
%   ``\(x=6\)''
% \end{quote}
% \begin{quote}
%   ``A limit of sums is a sum of limits.''
% \end{quote}
% there is an underlying logical structure that governs the meanings of such
% statements and how we conceptualize and work with them. For this reason, I have
% found discrete mathematics extremely enlightening in my own study of calculus
% and beyond.

This is now the longest document I've ever written. It has grown to represent
a sizable portion of my college education at this time. It's also the first time I've
developed a sustainable organizational system for my notes. Everything before
this, and everything besides this, lies in stacks of scattered legal pads in at
least four different locations.

My goal is to finally organize my thoughts and conceptualize this material in a
way I have never even attempted before.

Whatever it takes.

\hfill{Nathan Typanski}

\hfill \date{April 9, 2012}

\newpage

This text is a work in progress.

Everything in this document is subject to change.

No claim is made as to the accuracy of any of the information contained herein. There may be mistakes, inaccuracies, or outright lies included among otherwise relevant and complete content. Always check with a reputable source (e.g. a math book).
