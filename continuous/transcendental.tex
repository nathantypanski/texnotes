\chapter{Transcendental Functions}

\begin{defn}
  A \textbf{transcendental function} is a function that does not satisfy a polynomial equation whose coefficients are themselves polynomials, in contrast to an algebraic function, which does satisfy such an equation.
\end{defn}

In other words, a transcendental function is a function that ``transcends'' algebra in the sense that it cannot be expressed in terms of a finite sequence of the algebraic operations of addition, multiplication, and root extraction.

Examples of transcendental functions include the \emph{exponential function}, the \emph{logarithm}, and the \emph{trigonometric functions}.
%\cite{wiki:transcendental}

Formally,

\begin{defn}
  An analytic function \(f(z)\) of the real or complex variables \(z_1, \ldots, z_n\) is \textbf{transcendental} if the \(n+1\) functions \(z_1, \ldots, z_n\) are algebraically independent.
  \cite{wiki:transcendental}
\end{defn}

\section{Natural Logarithms}

\subsection{The Natural Logarithm and $e$}

\begin{defn}
	The \textbf{natural logarithm}is the function given by
  \begin{equation}
    \ln x = \int ^{x} _{1} \frac{1}{t} \ud t \text{,} \qquad x \in \mathbb{N}
  \end{equation}
\end{defn}
\begin{defn}
  The \textbf{number $e$} is that number in the domain of the natural logarithm satisfying
  \[ \ln{e}=1 \]
\end{defn}


\subsection{Algebraic Properties of the Natural Logarithm}

For any numbers $b>0$ and $x>0$, the natural logarithm satisfies the following rules:
\begin{table}
    \begin{tabular}{p{3in}>\(p{3in}<\)}
      Product Rule      & \displaystyle{ \ln{bx}=\ln b + \ln x} \\\\
      Quotient Rule     & \displaystyle{ \ln{\frac{b}{x}}=\ln b - \ln x} \\ \\
      Reciprocal Rule   & \displaystyle{ \ln{\frac{1}{x}}=-\ln x} \\\\
      Power Rule        & \displaystyle{ \ln{x^r}=r \ln x \qquad \forall r \in \mathbb{R}}
    \end{tabular}
\end{table}
% \begin{equation}
% 	\ln{bx}=\ln b + \ln x
% \end{equation}
% \begin{equation}
% 	\ln{\frac{b}{x}}=\ln b - \ln x
% \end{equation}
% \begin{equation}
% 	\ln{\frac{1}{x}}=-\ln x
% \end{equation}
% \begin{equation}
% 	\forall r \in \mathbb{R} \quad \ln{x^r}=r \ln x
% \end{equation}

\section{Hyperbolic Functions}
Both \(\cos x\) and \(\sin x\) come from the formula for a circle.
\begin{equation}
  x^2 + y^2=r^2
  \label{eq:circle}
\end{equation}

But we can define other useful functions using the equation for a hyperbola.
\begin{equation}
  x^2-y^2=1
  \label{eq:hyperbola}
\end{equation}
Namely, \(\cosh x\) and \(\sinh x\).

In \ref{eq:hyperbola}, let \[ y \to \frac{e^x-e^{-x}}{2}\] to get \(\sinh x\).
Let \[ x \to \frac{e^x+e^{-x}}{2}\] to find \(\cosh x\).

We can prove that these still satisfy equation \ref{eq:hyperbola}:

\begin{proof}
  \begin{align*}
    1&=x^2-y^2 \\
    1&=\left( \frac{e^x+e^{-x}}{2} \right) - \left( \frac{e^x - e^{-x}}{2}
    \right)^2 \\
    1&=\frac{e^{2x}+2e^xe^{-x}+e^{-2x}}{4}-\frac{e^{2x}-2e^xe^{-x}+e^{-2x}}{4}
    \qedhere
  \end{align*}
\end{proof}

\section{Inverse Functions}\index{inverse functions}

An inverse function undoes, or inverts, the effects of an original function.
While inverse functions are not, by definition, transcendental, we will look at them in this chapter because they are useful for producing inverse trigonometric functions--functions that are transcendental.
\begin{defn}
  Suppose that \(f\) is a one-to-one function on a domain $D$ with range $R$. The \textbf{inverse function} $f^{-1}$ is defined by
  \[ f^{-1}(b)=a \text{ if } f(a)=b \]
  The domain of $f^{-1}$ is $R$ and the range of $f^{-1}$ is $D$.
\end{defn}
In order for an inverse function $f^{-1}(x)$ to exist for a function $f(x)$, the original function $f(x)$ must be one-to-one. Otherwise, the resulting ``inverse function'' would not be a function: more than one output would be produced from only one input, and it would not pass the vertical line test.
\begin{remark} \index{composition}\index{inverse functions}
  By definition, either composite of a function and its inverse will return the identity function.
  \[ (f \circ f^{-1})(x)=f(f^{-1}(x))=x \]
  or
  \[ (f^{-1} \circ f)(x)=f^{-1}(f(x))=x \]
\end{remark}

\subsection{Finding Inverse Functions}
To find the inverse of a function $f(x)$, replace $f(x)$ with $y$ and solve for $x$ in terms of $y$. Then, interchange $x$ and $y$.
\begin{ex}
  Find the inverse of $y=\frac{1}{2}x+1$, expressed as a function of $x$.
  \begin{align*}
    y &= \frac{1}{2}x+1 \\
    2y &= x + 2 \\
    x &= 2y -2 \xrightarrow{\text{(swap $x$ and $y$)}} \\
    y &=2x-2
  \end{align*}
  The inverse of the function $f(x)=\frac{1}{2}x+1$ is the function $f^{-1}(x)=2x-2$.
\end{ex}



