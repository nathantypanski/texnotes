\chapter{Logic Circuits} \index{logic circuits}
\epigraph{
Look, everything we're putting into that box becomes ungrounded, and I don't
mean grounded like to the earth, I mean, not tethered. I mean, we're blocking
whatever keeps it moving forward and so they flip-flop. Inside the box it's like
a street, both ends are cul-de-sacs. I mean, this isn't frame dragging or
wormhole magic, this is basic mechanics and heat 101.}
{\emph{Primer}, 2004}

A logic circuit receives input signals \(p_1, p_2, \ldots, p_n\), each a bit, and produces output signals
\(s_1, s_2, \ldots, s_n\), each a bit.

\begin{figure}[h]
  \center{
      \subfigure[and]{
        \begin{circuitikz} \draw
          (0,2) node[and port] (myand1) {}
          (myand1.in 1) node[anchor=east] {}
          (myand1.in 2) node[anchor=east] {}
          (myand1.out) node[anchor=west] {};
        \end{circuitikz}
      }
      \subfigure[or]{
        \begin{circuitikz} \draw
          (0,2) node[or port] (myor1) {}
          (myor1.in 1) node[anchor=east] {}
          (myor1.in 2) node[anchor=east] {}
          (myor1.out) node[anchor=west] {};
        \end{circuitikz}
      }
      \subfigure[nor]{
        \begin{circuitikz} \draw
          (0,2) node[nor port] (mynor1) {}
          (mynor1.in 1) node[anchor=east] {}
          (mynor1.in 2) node[anchor=east] {}
          (mynor1.out) node[anchor=west] {};
        \end{circuitikz}
      }
      \subfigure[xor]{
        \begin{circuitikz} \draw
          (0,2) node[xor port] (myxor1) {}
          (myxor1.in 1) node[anchor=east] {}
          (myxor1.in 2) node[anchor=east] {}
          (myxor1.out) node[anchor=west] {};
        \end{circuitikz}
      }
      \subfigure[not]{
        \begin{circuitikz} \draw
          (0,2) node[not port] (mynot1) {}
          (mynot1.in) node[anchor=east] {}
          (mynot1.out) node[anchor=west] {};
        \end{circuitikz}
      }
  }
  \caption{Basic logic gates.}
\end{figure}

\begin{comment}
\begin{figure}[h]
  \begin{center}
    \begin{circuitikz}
      \draw
      (8, 2) node[and port] (and0) {}

      (3, 4) node[or port] (or0) {}
      (3, 0) node[or port] (or1) {}
      (5, 2) node[or port] (or2) {}

      (1, 4) node[not port] (not0) {}
      (2, 2) node[not port] (not1) {}
      (1, 0) node[not port] (not2) {}

      (not0.out) -- (or0.in 2)
      (not0.in) -- (0, 4) node[anchor=east] {\(p_2\)}
      (or0.in 1) -- (0, 5) node[anchor=east] {\(p_1\)}
      (not1.in) -- (0, 3) node[anchor=east] {\(p_3\)}
      (not2.in) -- (0, 0) node[anchor=east] {\(p_5\)}
      (or1.in) -- (0, 2) node[anchor=east] {\(p_4\)}
      (and0.out) -- (9,2) node[anchor=west] {\(s_1\)}
      (or1.out) -- (or2.in 2)
      (or0.out) -- (and0.in 1)
      (not1.out) -- (or2.in 1)
      (not2.out) -- (or1.in 2)
      (or2.out) -- (and0.in 2);
    \end{circuitikz}
  \end{center}
  \caption{A simple logic circuit}
\end{figure}
\end{comment}

