\chapter{Basics of Functions}
\section{Definition of a Function}

\begin{defn}
  A \textbf{function} \(f\) from a set \(D\) to a set \(Y\) is a rule that assigns a \emph{unique} (single) element \(f(x) \in Y\) to each element \(x \in D\).
  \index{function}
\end{defn}

\begin{defn}
  The \textbf{vertical line test} for a function is based on the idea that if \(a\) is in the domain of the function \(f\) then the vertical line \(x=a\) will intersect the graph of \(f\) at a single point \( \big(a,f(a)\big)\).
  \index{vertical line test}
\end{defn}

% One technique that can help us understand functions is by describing them with words, including mathematical language.
% We can discuss the behavior of a function in general, describing its properties, or we can discuss its behavior in more specific terms, like when we evaluate a function at a certain value.
%
% We can also talk about functions using visual language, as in graphs, arrow diagrams, and tables.
% There are, of course, many ways to visually represent a function, but for our purposes these will prove the most useful.
% This includes some numerical displays of functions, like with a table of values.
% We can consider tables to be a visual description of a function because they combines visual formatting with descriptive data to produce a figural representation of a function's properties.

We will find graphs, in conjunction with the respective mathematical language, to be the most cohesive way to represent functions.
We should therefore attempt to be able to graph our functions whenever we wish to achieve the best possible understanding of them.

We will combine these techniques in an effort to paint a complete picture of its behavior.

\section{Properties of Functions}


\subsection{Domain and Range}

\begin{defn}
  The set \(D\) of all possible input values is called the \textbf{domain} of the function.
  \index{domain}
\end{defn}

\begin{defn}
  The set of all values \(f(x)\) as \(x\) varies throughout \(D\) is called the \textbf{range} of the function, representing the output value of \(f\) at \(x\).
  \index{range}
\end{defn}

\begin{defn}
   The \textbf{natural domain} of a function is the largest set of real \(x\)-values for which a function returns real \(y\)-values.
   \begin{remark}
     When we define a funciton \(y=f(x)\) with a formula and the domain is not stated explicitly or restricted by context, \(D \to \mathbb{N}\).
   \end{remark}
   \index{natural domain}
\end{defn}

\begin{defn}
  A function is said to be \textbf{real-valued} when \(D \equiv \mathbb{R}\). That is, the domain is equivalent to the set of real numbers.
  \index{real-valued}
\end{defn}


\subsection{Dependent and Independent Variables}
\begin{defn}
  The letter \(x\) in the notation \(y=f(x)\) is called the \textbf{independent variable} of the function, representing the input value of \(f\).
  \index{independent variable}
\end{defn}
\begin{defn}
  The letter \(y\) in the notation \(y=f(x)\) is called the \textbf{dependent variable}.
  It varies with respect to change in the dependent variable of the function.
  \index{dependent variable}
\end{defn}

\subsection{Even and Odd Functions}

For a function $y=f(x)$, we describe its type of symmetry by calling the function \textbf{even}\index{even functions} or \textbf{odd}\index{odd functions}.

\begin{defn}
  An \textbf{even function} means $f(-x)=f(x)$. An example of this is the function $f(x)=x^2$.
  \begin{figure}[H]
    \begin{center}
      \begin{tikzpicture}
        \begin{axis}[
            ylabel={\(f(x)=x^2\)},
            axis x line=bottom,
            axis y line=center,
            tick align=outside,
            yticklabels={,,}
            xticklabels={,,}
            xtickmax=10,
          ]
          \addplot[smooth,red]{x^2};
        \end{axis}
      \end{tikzpicture}
    \end{center}
  \end{figure}
\end{defn}
\begin{defn}
  An \textbf{odd function} means $f(-x)=-f(x)$. An example of this is the function $f(x)=x^3$.
  \begin{figure}[H]
    \begin{center}
      \begin{tikzpicture}
        \begin{axis}[
            ylabel={\(f(x)=x^3\)},
            axis x line=bottom,
            axis y line=center,
            tick align=outside,
            yticklabels={,,}
            xticklabels={,,}
            xtickmax=10,
          ]
          \addplot[smooth,red]{x^3};
        \end{axis}
      \end{tikzpicture}
    \end{center}
  \end{figure}
\end{defn}
\subsection{Surjective, Injective, and Bijective Functions}

\begin{defn}
  \index{one-to-one}
  \index{injective}
  \( f: A \to B \) is \textbf{injective} (or \emph{one-to-one})
  \(
    \forall{(x_1 \wedge x_2 \in A)}
    \bigg[f(x_1)=f(x_2)
    \to x_1=x_2\bigg]
    \text{,}
  \)
  which also means that for injective functions,
  \( x_1 \neq x_2 \to f(x_1) \neq f(x_2)\).
\end{defn}
\begin{figure}[H]
    \begin{center}
        \subfigure[The function \(f(x)=x^2\) is not \emph{one-to-one} because there are two possible \(x\)-values that can produce any given \(y\)-value.]
        {
          \begin{tikzpicture}
            \begin{axis}[
                ylabel={\(f(x)=x^2\)},
                axis x line=bottom,
                axis y line=center,
                tick align=outside,
                yticklabels={,,}
                xticklabels={,,}
                xtickmax=10,
              ]
              \addplot[smooth,red]{x^2};
            \end{axis}
          \end{tikzpicture}
        }
        \hspace{0.2in}%
        \subfigure[The function \(f(x)=x^3\) is \emph{one-to-one} because any given \(y\)-value is mapped from a unique \(x\)-value.]
        {
          \begin{tikzpicture}
            \begin{axis}[
                ylabel={\(f(x)=x^3\)},
                axis x line=bottom,
                axis y line=center,
                tick align=outside,
                yticklabels={,,}
                xticklabels={,,}
                xtickmax=10,
              ]
              \addplot[smooth,blue]{x^3};
            \end{axis}
          \end{tikzpicture}
        }
    \end{center}
  \end{figure}
  \begin{remark}
    A function \(y=f(x)\) is one-to-one iff its graph intersects each horizontal line at most once.\index{horizontal line test}
  \end{remark}
\begin{defn}
  \index{onto}
  \index{surjective}
  \(f: A \to B \) is \textbf{surjective} (or onto) iff
    \(\forall{\left( b \in B \right)} \exists {\left( a \in A \right)} \big(f(a)=b\big) \).
\end{defn}
\begin{figure}[H]
    \begin{center}
        \subfigure[The function \(f(x)=x^2\) is not \emph{surjective} because the values \((-\infty, 0)\) are never reached in its range.]
        {
          \begin{tikzpicture}
            \begin{axis}[
                ylabel={\(f(x)=x^2\)},
                axis x line=bottom,
                axis y line=center,
                tick align=outside,
                yticklabels={,,}
                xticklabels={,,}
                xtickmax=10,
              ]
              \addplot[smooth,red]{x^2};
            \end{axis}
          \end{tikzpicture}
        }
        \hspace{0.2in}%
        \subfigure[The function \(f(x)=x^3\) is \emph{one-to-one} because all \(y\) values from \(-\infty, \infty)\) have corresponding \(x\)-values.]
        {
          \begin{tikzpicture}
            \begin{axis}[
                ylabel={\(f(x)=x^3\)},
                axis x line=bottom,
                axis y line=center,
                tick align=outside,
                yticklabels={,,}
                xticklabels={,,}
                xtickmax=10,
              ]
              \addplot[smooth,blue]{x^3};
            \end{axis}
          \end{tikzpicture}
        }
    \end{center}
  \end{figure}

  \begin{defn}
    \index{bijective}
    A function \(f:A \to B\) is \textbf{bijective} iff it is \emph{both injective and surjective}.
  \end{defn}
\begin{figure}[H]
    \begin{center}
        \subfigure[The function \(f(x)=x^2\) is not bijective.]
        {
          \begin{tikzpicture}
            \begin{axis}[
                ylabel={\(f(x)=x^2\)},
                xlabel={\(x\)},
                axis x line=bottom,
                axis y line=center,
                tick align=outside,
                yticklabels={,,}
                xticklabels={,,}
                xtickmax=10,
              ]
              \addplot[smooth,red]{x^2};
            \end{axis}
          \end{tikzpicture}
        }
        \hspace{0.2in}%
        \subfigure[The function \(f(x)=x^3\) is bijective.]
        {
          \begin{tikzpicture}
            \begin{axis}[
                ylabel={\(f(x)=x^3\)},
                xlabel={\(x\)},
                axis x line=bottom,
                axis y line=center,
                tick align=outside,
                yticklabels={,,}
                xticklabels={,,}
                xtickmax=10,
              ]
              \addplot[smooth,blue]{x^3};
            \end{axis}
          \end{tikzpicture}
        }
    \end{center}
  \end{figure}


\subsection{Graphs} \index{graphs}

\begin{defn}
  \index{graph}
  If \(f\) is a function with a domain \(D\), then its \textbf{graph} is
  \[ \Big\{ \big( x,f(x) \big) \Big | x \in D \Big\}\text{.}\]
\end{defn}

If \( (x,y) \) is a point on \(f\), then \(y=f(x)\) is the height of the graph above point \(x\).
This height might be positive or negative, depending on the sign of \(f(x)\).
\begin{figure}[H]
    \begin{center}
        \begin{tikzpicture}
          \begin{axis}[
              ylabel={\(f(x)\)},
              xlabel={\(x\)},
              axis x line=bottom,
              axis y line=center,
              tick align=outside,
              yticklabels={,,}
              xticklabels={,,}
              xtickmax=10,
            ]
            \addplot[smooth,red]{x+2};
          \end{axis}
        \end{tikzpicture}
      \caption{A graph of the function \(f(x)=x+2\)}
    \end{center}
  \end{figure}

\section{Composition of Functions}

\begin{defn}
  If \(f\) and \(g\) are functions, then the \textbf{composite} function \(f \circ g\) (``\(f\) composed with \(g\)'') is defined by
  \[ (f \circ g)(x)=f\bigl(g(x)\bigr) \text{.} \]
  \begin{remark}
    The domain of \( f \circ g \) consists of the numbers \(x\) in the domain of \(g\) for which \(g(x)\) lies in the domain of \(f\).
  \end{remark}
  \index{composition}
\end{defn}
\begin{ex}
  If
  \(f(x)=x^2\)
  and
  \(g(x)=1-\sqrt{x}\text{,}\)
  find \( (f \circ g)(x) \) and \( (g \circ f)(x)\).
  \begin{sol}
    \begin{align*}
      (f \circ g)(x)
      &=f\big(g(x)\big)
      &&&
      (g \circ f)(x)
      &=g\big(f(x)\big)=1-\sqrt{x^2}
      \\
      &=(1-\sqrt{x})^2
      &&&
      &= 1-|x|
    \end{align*}
  \end{sol}
\end{ex}
